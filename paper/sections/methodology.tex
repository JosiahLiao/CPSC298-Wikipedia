% Methodology
% Be specific enough that someone could reproduce your work

\label{sec:methodology}

% Overview
We retrieve data from Wikipedia's databases on users' edit history on Wikipedia articles and users' activity on article talk pages and the talk pages of other users. We then [perform some computational graphing analysis], helping us locate clusters of users who have connected activity in similar areas. By analyzing these connections, we can quantifiably determine what a "community" is on Wikipedia. We then analyze these communities to find common trends and similarities within and between communities.

\subsection{Data Collection}

[Describe what data you collected, from where, covering what time period, etc.]

% Example of how to cite a dataset or tool:
% We collected data from the Wikipedia API \cite{wikipedia-api} covering...

\subsection{Data Processing}

[Describe how you cleaned, filtered, or transformed your data]

% Example of code reference:
% Our data processing pipeline is available in our GitHub repository.\footnote{\url{https://github.com/yourusername/yourrepo}}

\subsection{Analysis Methods}

[Describe your analytical approach. What techniques did you use? Why?]

% If you have equations, you can include them:
% \begin{equation}
% your\_formula = here
% \end{equation}

\subsection{Ethical Considerations}
Optional: Include this if your project involves human subjects data, user behavior analysis, or if your institution requires ethics discussion. For basic Wikipedia article analysis, this may not be necessary - just ensure proper citation in your Data Collection section.

Example: All data consists of publicly available Wikipedia content accessed in compliance with \href{https://foundation.wikimedia.org/wiki/Policy:Terms_of_Use}{Wikimedia's Terms of Use}.

