% Related Work
% Organize by themes/categories, not paper-by-paper
% Show how your work builds on, differs from, or fills gaps in existing work

\label{sec:related}

% Introduction to related work
[Brief paragraph introducing the landscape of related research]

% Subsection 1: First category of related work
\subsection{Wikipedia Governance and Community Culture}

Reagle's seminal work on Wikipedia culture~\cite{reagle2010good} examines the collaborative practices and governance mechanisms that enable Wikipedia's success. An important work on community formation and user organization on Wikipedia comes from Rijshouwer~\cite{rijcomm}. In this work, Rijshouwer studies the formation of community culture throughout the history of Wikipedia. He then goes on to study the evolution of bureaucracy in Wikipedia's governance and how power becomes more centralized through natural organizational means as users create their own communities. Pentzhold's work on defining community in relation to Wikipedia as a sociological concept informs further discussions of what an online community is~\cite{pentz}. Pfeil, Zaphiris, and Ang's work studying cultural differences between different versions of Wikipedia uses content analysis to gain insight through user edits.

% Positioning your work
\subsection{Our Work in Context}

Exisiting research thus far has mainly focused on the Wikipedia community holistically. Much of the available research has focused on Wikipedia community as an abstract concept, or has focused on a specific aspect of Wikipedia community. Our work instead makes connections between Wikipedia sub-communities, using raw data analysis to remove any preconcieved notions of what makes a Wikipedia community.
