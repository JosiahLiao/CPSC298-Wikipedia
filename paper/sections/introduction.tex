% Introduction
% Structure: Problem/Motivation -> Background -> Research Questions -> Contributions -> Paper Outline

% Motivating example or opening
[Start with a compelling example or statement about Wikipedia governance]

% The problem/question
[What is the specific problem or question you're investigating?]

% Why it matters
[Why is this important? Who cares about this problem?]

% Research questions
This paper investigates the following research questions:
\begin{enumerate}
    \item Your first research question
    \item Your second research question
    \item Add more as needed
\end{enumerate}

% Contributions
The main contributions of this work are:
\begin{itemize}
    \item First contribution - e.g., novel analysis of X
    \item Second contribution - e.g., findings about Y
    \item Third contribution - e.g., methodology for Z
\end{itemize}

% Paper outline
The rest of this paper is organized as follows. 
Section~\ref{sec:related} reviews related work on Wikipedia governance. 
Section~\ref{sec:methodology} describes our data and methods.
Section~\ref{sec:results} presents our findings.
Section~\ref{sec:discussion} discusses implications and limitations.
Section~\ref{sec:conclusion} concludes and suggests future work.

