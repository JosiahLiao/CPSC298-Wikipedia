% Introduction
% Structure: Problem/Motivation -> Background -> Research Questions -> Contributions -> Paper Outline

% Motivating example or opening
In a minimally restricted environment, it is fascinating to study how people self-govern and create communities. Wikipedia is one such minimally restricted environment. Where many moderated or otherwise controlled online environments often devolve into chaos, Wikipedia has remained relatively stable with a mostly user-managed infrastructure. Without the guidance of higher powers, communities will almost inevitably form in such a stable social system. At first glance, these communities may seem simple to categorize. However, when performing deeper analysis we may discover fascinating or shocking connections within these communities that subvert our expectations for what makes a community.

% The problem/question
In this paper we seek to answer how communities are formed on Wikipedia and how the landscape of Wikipedia communities looks at this point in time. We look at what different communities revolve around, whether it is a subject matter, a set of articles, or a group of people. Through this we can make conclusions based on the connections between users in a free platform in this case.

% Why it matters
By studying Wikipedia communities we can analyze what has kept Wikipedia healthy and growing in a time where online discourse is increasingly divided and self-centered. Additionally, studying the formation of online societies provides fascinating insights into the fields of sociology and psychology, revealing connections we may not expect.

% Research questions
This paper investigates the following research questions:
\begin{enumerate}
    \item What are the main large communities on Wikipedia?
    \item What draws these communities together and sustains them?
    \item How do these communities interact with each other and the platform at large?
\end{enumerate}

% Contributions
The main contributions of this work are:
\begin{itemize}
    \item Observation of communities on Wikipedia
    \item Analysis of connections between Wikipedia communities
    \item Use of computational graphing for online sociological research 
\end{itemize}

% Paper outline
The rest of this paper is organized as follows. 
Section~\ref{sec:related} reviews related work on Wikipedia governance. 
Section~\ref{sec:methodology} describes our data and methods.
Section~\ref{sec:results} presents our findings.
Section~\ref{sec:discussion} discusses implications and limitations.
Section~\ref{sec:conclusion} concludes and suggests future work.

